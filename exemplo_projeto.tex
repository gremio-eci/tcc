%%
%% This is file `example.tex',
%% generated with the docstrip utility.
%%
%% The original source files were:
%%
%% coppe.dtx  (with options: `example')
%% 
%% This is a sample monograph which illustrates the use of `coppe' document
%% class and `coppe-unsrt' BibTeX style.
%% 
%% \CheckSum{1416}
%% \CharacterTable
%%  {Upper-case    \A\B\C\D\E\F\G\H\I\J\K\L\M\N\O\P\Q\R\S\T\U\V\W\X\Y\Z
%%   Lower-case    \a\b\c\d\e\f\g\h\i\j\k\l\m\n\o\p\q\r\s\t\u\v\w\x\y\z
%%   Digits        \0\1\2\3\4\5\6\7\8\9
%%   Exclamation   \!     Double quote  \"     Hash (number) \#
%%   Dollar        \$     Percent       \%     Ampersand     \&
%%   Acute accent  \'     Left paren    \(     Right paren   \)
%%   Asterisk      \*     Plus          \+     Comma         \,
%%   Minus         \-     Point         \.     Solidus       \/
%%   Colon         \:     Semicolon     \;     Less than     \<
%%   Equals        \=     Greater than  \>     Question mark \?
%%   Commercial at \@     Left bracket  \[     Backslash     \\
%%   Right bracket \]     Circumflex    \^     Underscore    \_
%%   Grave accent  \`     Left brace    \{     Vertical bar  \|
%%   Right brace   \}     Tilde         \~}
%%
\documentclass[grad,numbers]{coppe}
\usepackage{amsmath,amssymb}
\usepackage{hyperref}
\usepackage[utf8]{inputenc}
\usepackage[brazil]{babel}
\usepackage[T1]{fontenc}
\usepackage{graphicx}

\makelosymbols
\makeloabbreviations

\begin{document}
  \title{Título da Tese}
  \foreigntitle{Thesis Title}
  \author{Nome do Autor}{Sobrenome}
  \advisor{Prof.}{Nome do Primeiro Orientador}{Sobrenome}{D.Sc.}
  \advisor{Prof.}{Nome do Segundo Orientador}{Sobrenome}{Ph.D.}
  \advisor{Prof.}{Nome do Terceiro Orientador}{Sobrenome}{D.Sc.}

  \examiner{Prof.}{Nome do Primeiro Examinador Sobrenome}{D.Sc.}
  \examiner{Prof.}{Nome do Segundo Examinador Sobrenome}{Ph.D.}
  \examiner{Prof.}{Nome do Terceiro Examinador Sobrenome}{D.Sc.}
  \examiner{Prof.}{Nome do Quarto Examinador Sobrenome}{Ph.D.}
  \examiner{Prof.}{Nome do Quinto Examinador Sobrenome}{Ph.D.}
  
  
  
  \department{EET}% Confira a tabela a seguir para saber como preencher o comando \department de acordo com seu curso (Graduação - Poli) ou programa (Pós-Graduação - COPPE).
  
  %%%%%% Para alunos da POLI %%%%%%
  
  %% Course											Option
  %% Engenharia Ambiental                             EA
  %% Engenharia Civil                                 ECV
  %% Engenharia de Computação e Informação            ECI
  %% Engenharia de Controle e Automação               ECA
  %% Engenharia de Materiais                          EMAT
  %% Engenharia de Petróleo                           EPT
  %% Engenharia de Produção                           EPR
  %% Engenharia Eletrônica e de Computação            EEC
  %% Engenharia Elétrica                              EET
  %% Engenharia Mecânica                              EMC
  %% Engenharia Metalúrgica                           EMET
  %% Engenharia Naval e Oceânica                      ENO
  %% Engenharia Nuclear                               ENU
  
  
  %%%%%% Para alunos da COPPE %%%%%%
  
  %% Program											Option
  %% Engenharia Biomédica								PEB
  %% Engenharia Civil									PEC
  %% Engenharia Elétrica								PEE
  %% Engenharia Mecânica								PEM
  %% Engenharia Metalúrgica e de Materiais				PEMM
  %% Engenharia Nuclear									PEN
  %% Engenharia Oceânica								PENO
  %% Planejamento Energético							PPE
  %% Engenharia de Produção								PEP
  %% Engenharia Química									PEQ
  %% Engenharia de Sistemas e Computação				PESC
  %% Engenharia de Transportes							PET
  
  
  
  
  
  
  \date{01}{2016}

  \keyword{Primeira palavra-chave}
  \keyword{Segunda palavra-chave}
  \keyword{Terceira palavra-chave}

  \maketitle

  \frontmatter
  
  \makecatalog
  
  \dedication{A alguém cujo valor é digno desta dedicatória.}

  \chapter*{Agradecimentos}

  Gostaria de agradecer a todos.

  \begin{abstract}

  Apresenta-se, nesta tese, ...

  \end{abstract}

  \begin{foreignabstract}

  In this work, we present ...

  \end{foreignabstract}

  \tableofcontents
  \listoffigures
  \listoftables
  \printlosymbols
  \printloabbreviations

  \mainmatter
%  \doublespacing
  \chapter{Introdução}

  Segundo a norma de formatação de teses e dissertações do
  Instituto Alberto Luiz Coimbra de Pós-graduação e Pesquisa de
  Engenharia (COPPE), toda abreviatura deve ser definida antes de
  utilizada.\abbrev{COPPE}{Instituto Alberto Luiz Coimbra de Pós-graduação e Pesquisa de Engenharia}

  Do mesmo modo, é imprescindível definir os símbolos, tal como o
  conjunto dos números reais $\mathbb{R}$ e o conjunto vazio $\emptyset$.
  \symbl{$\mathbb{R}$}{Conjunto dos números reais}
  \symbl{$\emptyset$}{Conjunto vazio}
  
  Você deve selecionar seu curso de engenharia usando o comando \texttt{\textbackslash department\{Sigla\}} e no lugar de Sigla inserir a sigla referente ao seu curso de engenharia. A tabela \ref{tab:courses} relaciona as siglas dos cursos de engenharia da Escola Politécnica da Universidade Federal do Rio de Janeiro (POLI-UFRJ), enquanto que a tabela \ref{tab:programs} relaciona as siglas dos programas de pós graduação da COPPE.\abbrev{POLI-UFRJ}{Escola Politécnica da Universidade Federal do Rio de Janeiro}
  
  
  \begin{table}[h]
    \caption{Siglas dos cursos de engenharia da Escola Politécnica da UFRJ.}
    \label{tab:courses}
    \centering
    {\footnotesize
    \begin{tabular}{|c|c|}
      \hline
      Sigla & Curso\\
      \hline
      EA &  Engenharia Ambiental \\
      ECV & Engenharia Civil\\
      ECI & Engenharia de Computação e Informação \\
      ECA & Engenharia de Controle e Automação \\
      EMAT & Engenharia de Materiais\\
      EPT & Engenharia de Petróleo\\
      EPR & Engenharia de Produção\\
      EEC & Engenharia Eletrônica e de Computação\\
      EET & Engenharia Elétrica\\
      EMC & Engenharia Mecânica\\
      EMET & Engenharia Metalúrgica\\
      ENO & Engenharia Naval e Oceânica\\
      ENU & Engenharia Nuclear\\
      \hline
    \end{tabular}}
    \end{table}
    
    
  \begin{table}[h]
  	\caption{Siglas dos programas de pós graduação da COPPE.}
  	\label{tab:programs}
  	\centering
  	{\footnotesize
  	\begin{tabular}{|c|c|}
  		\hline
  		Sigla & Curso\\
  		\hline
  		PEB & Engenharia Biomédica \\
  		PEC & Engenharia Civil\\
  		PEE & Engenharia Elétrica \\
  		PEM & Engenharia Mecânica \\
  		PEMM & Engenharia Metalúrgica e de Materiais\\
  		PEN & Engenharia Nuclear\\
  		PENO & Engenharia Oceânica\\
  		PPE & Planejamento Energético\\
  		PEP & Engenharia de Produção\\
  		PEQ & Engenharia Química\\
  		PESC & Engenharia de Sistemas e Computação\\
  		PET & Engenharia de Transportes\\
  		\hline
  	\end{tabular}}
  \end{table}


  Note também que todas as figuras ou tabelas devem ser citadas no texto. Como ocorre com as tabelas \ref{tab:courses} e \ref{tab:programs}. Para ilustrar o uso de figuras em \LaTeX, considere as figuras \ref{fig:poli} e \ref{fig:coppe}.
  
   \begin{figure}
      \centering
      \includegraphics[width=5cm]{poli-logo.pdf}
      \caption{Logotipo da POLI-UFRJ.}
      \label{fig:poli}
    \end{figure}
    
    \begin{figure}
       \centering
       \includegraphics[width=5cm]{coppe-logo.pdf}
       \caption{Logotipo da COPPE.}
       \label{fig:coppe}
     \end{figure}

  \chapter{Revisão Bibliográfica}

  Para ilustrar a completa adesão ao estilo de citações e listagem de
  referências bibliográficas, a Tabela \ref{tab:citation} apresenta citações de alguns dos trabalhos contidos na norma fornecida pela CPGP da
  COPPE, utilizando o estilo numérico.

  \begin{table}[h]
  \caption{Exemplos de citações utilizando o comando padrão
    \texttt{\textbackslash cite} do \LaTeX\ e
    o comando \texttt{\textbackslash citet},
    fornecido pelo pacote \texttt{natbib}.}
  \label{tab:citation}
  \centering
  {\footnotesize
  \begin{tabular}{|c|c|c|}
    \hline
    Tipo da Publicação & \verb|\cite| & \verb|\citet|\\
    \hline
    Livro & \cite{book-example} & \citet{book-example}\\
    Artigo & \cite{article-example} & \citet{article-example}\\
    Relatório & \cite{techreport-example} & \citet{techreport-example}\\
    Relatório & \cite{techreport-exampleIn} & \citet{techreport-exampleIn}\\
    Anais de Congresso & \cite{inproceedings-example} &
      \citet{inproceedings-example}\\
    Séries & \cite{incollection-example} & \citet{incollection-example}\\
    Em Livro & \cite{inbook-example} & \citet{inbook-example}\\
    Dissertação de mestrado & \cite{mastersthesis-example} &
      \citet{mastersthesis-example}\\
    Tese de doutorado & \cite{phdthesis-example} & \citet{phdthesis-example}\\
    \hline
  \end{tabular}}
  \end{table}
  
  É importante notar que, segundo a \href{http://www.poli.ufrj.br/graduacao_projeto.php}{Norma para a Elaboração Gráfica do Projeto de Graduação} da Escola Politécnica da UFRJ para trabalhos de conclusão de curso de engenharia de julho de 2012, as referências bibliográficas podem ser apresentadas de duas formas: $(i)$ Referências numeradas e $(ii)$ Referências em ordem alfabética. Para exibição numerada, em que a exibição das referências bibliográficas segue a ordem de citação usada no texto, use o comando \texttt{\textbackslash bibliographystyle\{coppe-unsrt\}}. Para exibição de referências bibliográficas em ordem alfabética, basta usar o comando \texttt{\textbackslash bibliographystyle\{coppe-plain\}} ao final do documento. 
  
  
  \chapter{Método Proposto}
  
  
  
  \chapter{Resultados e Discussões}
  
  
  
  \chapter{Conclusões}
  
  
  
  

  \backmatter
  \bibliographystyle{coppe-unsrt}
  \bibliography{example}

  \appendix
  \chapter{Algumas Demonstrações}
\end{document}
%% 
%%
%% End of file `example.tex'.
